\documentclass[OPS,authoryear,toc]{lsstdoc}
% lsstdoc documentation: https://lsst-texmf.lsst.io/lsstdoc.html
\input{meta}

% Package imports go here.

% Local commands go here.

%If you want glossaries
%\input{aglossary.tex}
%\makeglossaries

\title{Compute Resource Usage of DP0.2 production run}

% Optional subtitle
% \setDocSubtitle{A subtitle}

\author{%
Brian Yanny
}

\setDocRef{RTN-039}
\setDocUpstreamLocation{\url{https://github.com/lsst/rtn-039}}

\date{\vcsDate}

% Optional: name of the document's curator
% \setDocCurator{The Curator of this Document}

\setDocAbstract{%
Summary of cpu-time, memory, storage of DP0.2 production run in iDF.
}

% Change history defined here.
% Order: oldest first.
% Fields: VERSION, DATE, DESCRIPTION, OWNER NAME.
% See LPM-51 for version number policy.
\setDocChangeRecord{%
  \addtohist{1}{YYYY-MM-DD}{Unreleased.}{Brian Yanny}
}


\begin{document}

% Create the title page.
\maketitle
% Frequently for a technote we do not want a title page  uncomment this to remove the title page and changelog.
% use \mkshorttitle to remove the extra pages

% ADD CONTENT HERE
% You can also use the \input command to include several content files.

\appendix
% Include all the relevant bib files.
% https://lsst-texmf.lsst.io/lsstdoc.html#bibliographies
\section{References} \label{sec:bib}
\renewcommand{\refname}{} % Suppress default Bibliography section
\bibliography{local,lsst,lsst-dm,refs_ads,refs,books}

% Make sure lsst-texmf/bin/generateAcronyms.py is in your path
\section{Acronyms} \label{sec:acronyms}
\addtocounter{table}{-1}
\begin{longtable}{p{0.145\textwidth}p{0.8\textwidth}}\hline
\textbf{Acronym} & \textbf{Description}  \\\hline

B & Byte (8 bit) \\\hline
BPS & Batch Production Service \\\hline
CCD & Charge-Coupled Device \\\hline
DC & Data Center \\\hline
DESC & Dark Energy Science Collaboration \\\hline
DM & Data Management \\\hline
DP0 & Data Preview 0 \\\hline
DRP & Data Release Production \\\hline
GB & Gigabyte \\\hline
LSST & Legacy Survey of Space and Time (formerly Large Synoptic Survey Telescope) \\\hline
MB & MegaByte \\\hline
OPS & Operations \\\hline
PB & PetaByte \\\hline
POSIX & Portable Operating System Interface \\\hline
PanDA &  Production ANd Distributed Analysis system \\\hline
RAM & Random Access Memory \\\hline
RTN & Rubin Technical Note \\\hline
S3 & (Amazon) Simple Storage Service  \\\hline
TB & TeraByte \\\hline
bps & bit(s) per second \\\hline
\end{longtable}

% If you want glossary uncomment below -- comment out the two lines above
%\printglossaries





\end{document}
