\documentclass[OPS,authoryear,toc]{lsstdoc}
% lsstdoc documentation: https://lsst-texmf.lsst.io/lsstdoc.html
\input{meta}

% Package imports go here.

% Local commands go here.

%If you want glossaries
%\input{aglossary.tex}
%\makeglossaries

\title{Compute Resource Usage of DP0.2 production run}

% Optional subtitle
% \setDocSubtitle{A subtitle}

\author{%
Brian Yanny, Nikolay Kuropatkin, Huan Lin, Jennifer Adelman-McCarthy, Colin Slater, Hsin-Fang Chiang
}

\setDocRef{RTN-039}
\setDocUpstreamLocation{\url{https://github.com/lsst/rtn-039}}

\date{\vcsDate}

% Optional: name of the document's curator
% \setDocCurator{The Curator of this Document}

\setDocAbstract{%

We present a summary of wallclock and cpu time, memory and storage usage of 
the DP0.2 production run in the iDF.  Over the course 
of 150 calendar days in 2021-2022, preliminary Data Release 
Production (DRP) pipelines ran on 20,000 exposures of simulated Rubin data, 
representing 10-20 full nights of images, covering 
150 tracts (300 square degrees), to typically 5 year depth 
(50 returns to the same spot on the sky for each of 6 filters).  
This 'half-percent' survey used a compute cluster of approximately 
4000 cores, each core with up to 14 GB/core RAM 
available (less typically used) with jobs taking typically 
a few seconds to a fraction of an hour.  
Small subsets (a few percent) of the processing pipetasks used nodes with 
up to 236 GB/core, and were allowed to run for up to 3 days.  The 
input images were approximately 260TB in size, while the outputs, counting transient datasetTypes took up 3.4PB in an object store.  After removing transient
datasetTypes (e.g. warps), the final storage used by DP0.2 was approximately
2.5 PB.  Of the 150 calendar days, approximately 60 days were running and
the remainder used for debugging.  Thus, to keep up with expected DRP during
the course of the survey (6-12 days to process 10-20 nights of data), 
a factor of 5-10 speed up (or increased core counts, or some combination
of both) are estimated to be needed.
}

% Change history defined here.
% Order: oldest first.
% Fields: VERSION, DATE, DESCRIPTION, OWNER NAME.
% See LPM-51 for version number policy.
\setDocChangeRecord{%
  \addtohist{1}{YYYY-MM-DD}{Unreleased.}{Brian Yanny}
}


\begin{document}

% Create the title page.
\maketitle
% Frequently for a technote we do not want a title page  uncomment this to remove the title page and changelog.
% use \mkshorttitle to remove the extra pages

% ADD CONTENT HERE
% You can also use the \input command to include several content files.

\section{Introduction and Scope of DP0.2}

DP0.2 (Data Preview 0.2) consisted of the (re)processing of a 
set of approximately 20,000 Rubin camera simulated exposures (each with up to
189 detectors per exposure) covering 157 tracts 
(300 square degrees) in 6 filters ($ugrizy$) to a typical 
depth of half that of the full survey.  These 20K visits represent 
approximately 10 nights of data gathering, roughly 0.5\% of the 
full survey.

Processing of the DESC-generated DC-2 simulated dataset \citep{DESC} took place between Dec 18, 2021 and May 16, 2022.  Processing took place in a Google Cloud environment, named the iDF (Interim Data Facility).  In excess of 4,000 cores
were available in a number of queues and memory/core configurations
within the cloud production environment.  The total cpu usage over the 
course of DP0.2 was approximately 2.5M core-hours. Storage was provided 
in the form of S3 addressable object storage space.  At peak, the production 
used 3.3 PiB of object store. After removal of intermediate data products
(Warp images and intermediate single epoch CCD images), the final storage
in the S3 data store at the end of DP0.2 production amounted to approximately
2.55 PiB. 

The scope of production consisted of running seven pipeline steps. Each step
was in turn made up of one or more pipetasks.   Before production
started, the 20K raw exposures/visits were pre-loaded into S3 storage, along
with dimension and URI metadata in a common butler registry, 
used for all processing.  

A Skymap and Photometric solutions for the 
raw exposures were also provided ahead of time -- these were not created
as part of the DP0.2 processing.

After processing was complete, a copy of the large butler registry was made
for end user access and HiPS (large-format zoomable, pannable) image maps 
of the tracts were generated. Also, the detected and measured output object 
and source tables were put into a qServ database for collaboration
access.  These Butler copy, HiPS creation and qServ loading operations
are not included in the compute resources tabulated here. 

\section{Software Configuration}

The versions of the LSST software environment used for production were
based on {\bf v23} of the lsst distribution.  After the start of step 1 
processing a number of updates to
the software occured to address issues encountered during processing
and this resulted in the small revision numbers of the {\bf v23} distribution
to change during DP0.2.

The workflow software system was \cite{PanDA}, and the 
BPS (Batch Processing Service) software 
(part of the LSST distribution) was used to interface between the Butler 
and PanDA using a set of {\bf yaml} job submission files 
(one for each group of each step of processing, see below). In addition, the Google Cloud logging system was
used to collect logging information output by the production pipelines and
organize it into a searchable form sorted by date of production.

\section{Processing Groups and clustering directives}

Each step of the seven step processing flow for DP0.2 
(modeled on the eventual Data Release Production (DRP) system 
planned for regular Rubin processing) consists of a well
defined, ordered list of pipeline tasks (pipetasks) run on either a set of 
single or dual exposure visits (steps 1,2,4,6) or on 
coadd tracts (steps 3, 5, 7) (and their associated visits).  

The BPS system takes as input a yaml file. The yaml files used during DP0.2 are archived at: \url{https://github.com/lsst-dm/dp02-processing/tree/main/full/production}.  As step 1 production commenced it was determined that an arbitrarily
large number of quanta could not be successfully processed with a single
BPS submission.  For example, step 1 production runs five pipetasks on 
20K visits, where each visit consists of typically 150 detector images,
thus resulting in 20K*150*5 = 15M individual quanta of processing for
step 1.  

A key component of the PanDA workflow system is the "Intelligent Data Delivery Service" (IDDS) which manages the scheduling of individual processing
quanta to a set of processing cores. The BPS and step description yaml configurations guide the creation of a "quantum graph" which fully describes the job flow
dependencies (what tasks depend on what other tasks being complete).  Also
created at the same time as the quantum graph for each 'BPS submit', is a
so-called 'execution butler' which contains file metadata and 
URI information for all inputs needed to completely run a BPS processing submission.

While one can define a single large DP0.2 step 1 BPS configuration yaml file
(specifying all 20K visits to be processed), time and memory restrictions
on the Butler, PanDA and compute systems prevent successful completion of 
the implied 15M quanta processing from a single BPS submission.

One issue encountered was that the time to generate the quantum graph 
and execution butler was prohibitive, 
it would still take days and many hundreds of GB to generate such 
quantum graphs and execution butlers.  Practically speaking, 
there is a limit of a few hours and a few GB for preparation of and size of 
the quantum graph and execution butler for a single BPS submission.  

\subsection{Clustering of pipetasks}
A second limitation on how large a workflow can be accomodated in one
BPS submission comes with the operation of the IDDS and how
pipetasks are clustered (or not clustered).  The IDDS system has within 
it an $N^2$ algorithm which matches jobs done to jobs-to-be-done and 
determines which jobs may be run next, having their inputs already produced
as outputs from the previous step.

To use the Butler and PanDA workflow system in practice, some division of
sets of pipetasks into groups is done.  Also, clustering sets of related, 
one-after-the-other pipetasks (in step 1, for example the five pipetasks are 
run on each exposure/detector in this order: {\bf isr} followed by 
{\bf characterizeImage} followed by {\bf calibrate} followed by 
{\bf writeSourceTable} followed by  {\bf transformSourceTable}).  Since this
dependence structure is fixed and invariant it helps to reduce the load 
on IDDS if these five pipetasks are clustered into one pipetask, 
called {\bf visit$\_$focal$\_$plane$\_$1}.  While IDDS generally 
checks all running jobs and tries to match up any done job and its 
outputs to any requested 
waiting-to-run job with inputs that match the outputs aleady completed, 
having some ordered lists of pipetasks combined into a clustered task
allows IDDS to not have to match everything against everything in these
cases.  There are also savings in complexity in generating the
quantum graph in these cases.
Since the order and set of pipetasks is fixed and every output always 
flows to the next pipetask's input, IDDS need not try to match every 
possible pipetask {\bf B} to follow on from pipetask {\bf A}, thus greatly 
reducing the number of possible combinations panDA IDDS needs to 
examine to determine which quanta with which
inputs to run next through which job.  The named definition of clusters 
of pipetasks used in DP0.2 processing are 
given in {\bf clustering.yaml} files in 
\url{https://github.com/lsst-dm/dp02-processing/tree/main/full/production} 
for each step.  

\section{PanDA queues}
BPS submissions in DP0.2 ran in the Google Cloud, and a set of six panDA job 
queues weere established with different retry characteristics and 
memory limits.  The cost to use computing in the cloud increases 
linearly with maximum memory allocated per job and also computing is a 
factor of several times cheaper if the queue could 
be 'preempted' (forcing the job to run again from the start).  
These queues are listed in Table \ref{tab:queues}. Due to this 
structure, the cheapest queue is the pre-emptable lowest max 
memory queue, and the most expensive is the 
non-preemptable highest max memory queue. The default queue for BPS submits 
is the {\bf GOOGLE$\_$TEST} (pre-emptable) queue and the vast majority 
of jobs completed successfully in this queue.  The maximum number of
retries allowed was set by default to be 5, and the usual reason for a 
job to be retried was preemption.  If a job failed exactly the same 
way 5 times typically this indicated a pipeline problem, 
and these cases were fed back to the pipeline development team for investigation. 
If a long running job failed 5 times at different points in 
the processing, this often indicates rather a repeated preemption, 
and suggests a {\bf rescue} submission may be needed running with a 
higher max memory or in a non-preempt queue.

The special queue, GOOGLE$\_$MERGE, is used for merging output metadata 
and objects (image files, catalogs) back into the main Butler registry (for
metadata) with links to these objects recorded in the object store portion
of the Butler. 

\begin{center}
\begin{table}[ht]
\caption{PanDA queues used for DP0.2 processing}
\begin{tabular} { |l|r|r|l|}
\hline
queue & maxMem(GB) & used by &  Note\\
\hline
$\rm DOMA\_LSST\_GOOGLE\_TEST$ & 14 & default &\\
$\rm DOMA\_LSST\_GOOGLE\_MERGE$ & 14 & butler merge &\\
$\rm DOMA\_LSST\_GOOGLE\_HIMEM$ & 40 &  &\\
$\rm DOMA\_LSST\_GOOGLE\_HIMEM\_NON\_PREEMPT$ & 40 & &\\
$\rm DOMA\_LSST\_GOOGLE\_EXTRA\_HIMEM$ & 236 &  &\\
$\rm DOMA\_LSST\_GOOGLE\_EXTRA\_HIMEM\_NON\_PREEMPT$ & 236 & &\\
\hline
\end{tabular}
\label{tab:queues}
\end{table}
\end{center}


\section{Memory usage}

A few pipetasks required more than the usually amount of memory, and
these tasks were assigned to a higher max memory queue.  Also, a few 
related Q/A pipetasks occasionally required exceedingly long runtimes on
selected quanta, these were handled by a rescue BPS submissions,
each with their own bps submit yaml which indicated to the butler to only run
on quanta which did not previously complete, and indicated to panDA to run
in a special non-preemptable (NO PE) queue. The pipetasks which required 
extra memory or no-preempt status for at least some quanta are listed in
Table \ref{tab:maxmem}.  Also listed in Table \ref{tab:maxmem} are typical
maximum memory requirements for most quanta running through an indicated
pipetask.
\begin{center}
\begin{table}[ht]
\caption{Memory requirements for processing pipetasks}
\begin{tabular} { |l|r|r|}
\hline
pipetask & reqMem(MB) &  Notes\\
\hline
  makeWarp&  8192&\\
  assembleCoadd&  16384&\\
  deblend&  16384&\\
  measure&  16384&\\
  mergeMeasurements&  16384&\\
  forcedPhotCoadd&  4096&\\
  writeObjectTable&  16384&\\
  matchCatalogsPatch&  4096&\\
  modelPhotRepGal4&  4096&\\
  imageDifference&  4096&\\
  matchCatalogsPatchMultiBand&  120000&\\
  matchCatalogsTract&  120000&\\
  wPerp&  120000&\\
  TE1&  46000&\\
  TE2&  46000&\\
  consolidateObjectTable&  4096&\\
  consolidateForcedSourceTable&  120000&\\
  consolidateForcedSourceOnDiaObjectTable&  18000&\\
\hline
 deblend& 32768& step 3 rescue\\
  measure& 16384 & 40 GB NO PE \\
\hline
deblend& 32768& step 3 faro rescue \\
deblend& 32768& 236GB NO PE\\
matchCatalogsPatchMultiBand& 120000& 236GB NO PE\\
matchCatalogsTract& 120000 & 236GB NO PE\\
\hline
consolidateForcedSourceOnDiaObjectTable& 18000& step 5 40 GB\\
\hline
\end{tabular}
\label{tab:maxmem}
\end{table}
\end{center}


\section{Storage}

Storage for DP0.2 inputs and outputs was done using a large S3 compatible
(key= unique POSIX-style path+filename plus bucket prefix, value=pointer
to bytes of a data file or table) object store in the Google Cloud.   
The datasetType names for the types of objects which used the bulk 
of the storage are listed in Table \ref{tab:storage}.  Also listed
for each datasetType are estimates of the number of DP0.2 objects of 
a given type, the typical size of a single object of 
that type (in MB), the total DP0.2 storage for objects of 
that type (in TB), and which step in the processing
creates objects of that type.  Also, if a datasetType is transient, and 
is not retained once the downstream pipetasks which make use of that
datasetType are complete, the datasetType is marked with a 'D' in the 'Delete'
column.  The warp images are the largest example of datasetTypes that are
transient.
To illustrate, important retained datasetType are {\bf calexp} objects, 
representing a single (exposure,detector) corrected and 
calibrated ccd image.
Each {\bf calexp} takes about 101 MB of space, and there are 2.8M 
in the DP0.2 dataset (20,000 exposures times 150 detectors/exposure, on average, gives 3M as a rough estimate of the number of CCD images in the DP0.2 raw
input dataset).  Note that which the full LSST camera has 189 
detectors, for DP0.2, on average only 150 of these are present per exposure,
though some exposures do have all 189 detectors present in DP0.2.
Recall that in rough approximation for some subset of data production
and data products,  DP0.2 represents one-half of one percent 
(1/200th) of the full LSST.


\begin{center}
\begin{table}
\caption{Storage by datasetType}
\begin{tabular} { |r|r|r|r|r|r|}
\hline
TB& Nobj & MB/obj & step & D & dataSetType \\
\hline
260 & 2.9M & 90 & input & & raw \\
\hline
506&5.1M&104&step3&D&deepCoadd$\_$directWarp\\
506&2.2M&230&step4&&	goodSeeingDiff$\_$templateExp\\
389&5.1M&80&step3&D&deepCoadd$\_$psfMatchedWarp\\
293&2.9M&102&step1&D&icExp\\
283&2.8M&101&step1&&	calexp\\
258&2.9M&90&step1&D&postISRCCD\\
224&2.2M&102&step4&&	goodSeeingDiff$\_$differenceExp\\
200&3.4M&60&step4&&	forced$\_$diff\\
184&4.6M&40&step5&&	forced$\_$diff$\_$diaObject\\
150&3.4M&44&step4&&	mergedForcedSource\\
39&0.6M&651&step3&&	deepCoadd$\_$meas\\
18&4.6M&4&step5	&&forced$\_$src$\_$diaObject\\
12&7.5K&1600&step5&&	forcedSourceTable\\
12&0.6M&196&step3&&	deepCoadd\\
12&0.6M&196&step3&&	deepCoadd$\_$calexp\\
10&2.6M&3.7&step6&&	calibratedSource\\
8.4&	2.8M&3&step1&&	source\\
8.3&	0.6M&141&step3&&	goodSeeingCoadd\\
5 &4.6M&1.2&step5&&	mergedForcedSourceOnDiaObject\\
3.4&3.4M&1.0&step4&&	forced$\_$src\\
3.4&2.8M&1.20&step1&&	sourceTable\\
%\hline
%%\end{tabular}
%\label{tab:storage1}
%\end{table}
%\end{center}
%
%\begin{center}
%\begin{table}
%\caption{Storage (cont)}
%\begin{tabular} { |r|r|r|r|r|r|}
%\hline
%TB&Nobj&MB/Obj&step&D& DataSetType (continued)\\
%\hline
2.4&2.8M&0.85&step1&&	icSrc\\
1.6&10K&170&step3&&	objectTable\\
1.3&2.6M&0.5&step6&&	calibratedSourceTable\\
1&20K&50&step6	&&calibratedSourceTable$\_$visit\\
1.2&20K&58&step2&&	sourceTable$\_$visit\\
1.0&156&6600&step3&&	objectTable$\_$tract\\
0.5&60K&6.4&step3&&	deepCoadd$\_$inputMap\\
0.4&2.2M&0.18&step4&&	goodSeeingDiff$\_$diaSrc\\
0.3&2.8M&0.11&step1&&	calexpBackground\\
0.28&7.5K&36&step5&&	forcedSourceOnDiaObjectTable\\
0.2&150&1900&step5&&	forcedSourceOnDiaObjectTable$\_$tract\\
0.2&2.9M&0.08&step1&&	icExpBackground\\
0.2&60K&4&step3	&&deepCoadd$\_$det\\
0.2&60K&3&step3	&&deepCoadd$\_$measMatchFull\\
0.12&2.2M&0.06&step4&&	goodSeeingDiff$\_$diaSrcTable\\
0.1&60K&2.4&step3&&	deepCoadd$\_$nImage\\
0.1&60K&1.4&step3&&	goodSeeingCoadd$\_$nImage\\
0.02&20K&1&step6&&	diaSourceTable\\
0.02&2.8M&0.01&step1&&	srcMatch (src:7MB)\\
0.02&20K&0.10&step2&&	visitSummary\\
%\hline
%0.4& 200 & 2000 & BPS &   & Exec Butler\\
%0.22& 200 & 1100 & BPS &   & qgraph \\
\hline
3.4PB(1.9PB)& & & & & Total in S3 without(with) Deletes\\
\hline
\end{tabular}
\label{tab:storage}
\end{table}
\end{center}


\section{Processing Summary}

DP0.2 processing took place in 2021-2022, over the course of about 150
calendar days.  Processing was done step-by-step, using the 7 pipetask
sets defined by the pipeline team in the v23 release of the processing
software distribution.  Work was done to break the exposures or tracts
needed to be processed into managable {\bf groups}.  The size of each
group in terms of numbers of exposures or tracts was set so that no
quantum graph or execution butler generation took more than about 1
hour per group, and so that no group's total run time exceeded
approximately 1 day (24 hours).  
Additionally, the panDA directive:
{\rm maxJobsPerTask: 30000} was added to some {\bf step's} BPS submit yaml
files to allow panDA to break groups with large number of quanta into
sets of no more than 30,000.  This allowed the IDDS job matching
algorithm, which determines what outputs are ready to be fed in as
inputs to the next pipetask in a set, to avoid taking excessive
memory, as the memory usage and compute time scale as $N^2$ where $N$
is the number of jobs in the task.  The limit of 30,000 allowed panDA
IDDS to run on a machine with 12GB RAM ($30000^2 \sim 1G$).  
Some trial-and-error was required to determine the optimal group 
size, in visits (steps 1,2, 4) or tracts (steps 3,5,6,7), for each {\bf step}.  
Most quanta in a group ran for similar lengths of time and used similar
amounts of memory.  There were, however, cases where a small percentage
of quanta exceeded the default memory of a queue and hung, or were repeatedly
pre-empted as they ran too long (sometimes for days), some of the Quality
Assurance metric tasks fell into this category.  For these
cases, nearly all in step 3 or step 5 processing, rescue BPS workflow
yamls were designed and submitted to higher max memory queues or 
to non-preempt queues to allow the subset of failed quanta to have the
resources needed to complete.

\subsection{DP0.2 Campaign tracking}

The preliminary operations campaign tools available in the {\bf
prodstatus} repo \url{https://github.com/lsst-dm/prodstatus} were
used to monitor DP0.2 processing and record compute resource metrics in 
a series of JIRA tickets, summarizing information about each group 
run for each of the seven DRP steps.   The original ticket 
requesting DP0.2 DRP processing is 
\url{https://jira.lsstcorp.org/browse/PREOPS-905}, which contains 
extensive comments about issues that came up during processing. 
A document describing the DP0.2 campaign in some detail is available 
as \url{https://rtn-041.lsst.io/v/DM-17130/index.html}.

Table \ref{tab:summarycpu} shows, for each step, 1) whether the step is
exposure(visit) or tract based, 2) the number of groups the exposures or
tracts are broken into, 3) the starting and ending calendar dates 
for processing, 4) the approximate number of core-hours used, and 
5) a software version stamp. 
In total, about 2.5M core-hours were used (on a cluster with about
4000 cores) over 150 calendar days.  Of these 150 days, approximately
60 days were used for running, with most of the remainder used for
debugging, rescue design, and group sizing trial-and-error.  
2.5M/4000/24 = 26 days, so cores were used at less than 50\% efficiency.

The input dataset represents data typically gathered over 10-20
observing nights.  DRP is required to process (not considering cumulative
reprocessing here) about 300 nights of data in 200 days, 
and so, 10-20 nights should be turned around in 6-12
calendar days.  These very approximate figures can be used to estimate rough
factors by which the processing needs to be sped-up or the factor by
which the number (or efficiency) of cores needs to be increased 
(or some combination of the two).  Roughly a factor of 5-10 speed 
up or size up is estimated to be needed.


\begin{center}
\begin{table}[ht]
\caption{Summary of DP0.2 cpu usage by step}
\begin{tabular} { |c|r|r|c|c|r|l|l|}
\hline
step & group by & N groups & start & end & core-hr & software & Note\\
\hline
step1 & visit/det & 14x3 &2022-12-18& 2022-01-12 & 166K & $\rm v23\_0\_0\_rc5$ &\\
step2 & visit  & 14 & 2022-01-20& 2022-01-24 & 22K & $\rm v23\_0\_0\_rc5$ &\\
step3 & tract & 33 & 2022-02-18& 2022-03-25 & 1100K & $\rm v23\_0\_1$ &\\
step4 & visit & 40& 2022-04-01& 2022-04-30 & 1100K & $\rm v23\_0\_1\_rc4$ &\\
step7 & all & 1 & 2022-05-01& 2022-05-01 & 10 & $\rm v23\_0\_2\_rc2$ &\\
step5 & tract & 52 & 2022-05-03& 2022-05-12 & 66K & $\rm v23\_0\_2\_rc2$ &\\
step6 & visit & 20&2022-05-12& 2022-05-16 & 16K & $\rm v23\_0\_2\_rc3$ &\\
purge & exp/warp & 2 & 2022-04-01& 2022-05-30 & 10 & $\rm v23\_0\_2\_rc2$ & \\
\hline
total & & & 150d & & 2500K &&\\
\hline
\end{tabular}
\label{tab:summarycpu}
\end{table}
\end{center}

The JIRA issues which record the {\bf BPS} submit yamls for each group
within each of the {\bf steps} of {\rm prodstatus} tracking of DP0.2
processing are linked from JIRA issue
\url{https://jira.lsstcorp.org/browse/DRP-473}.  From this 
{\bf campaign-level} issue, one can access the seven individual {\bf step-level}
issues, and from each of these in turn, the individual BPS submit
workflow groups can be accessed, with one JIRA issue per BPS submission. 
Several hundred BPS submissions make up the complete DP0.2 campaign.
For each of these, information on the number of quanta, 
the time-to-run, the max memory used per pipetask among 
other metrics are drawn from Butler and PanDA tables and are 
recorded along with a uniquely-identifying BPS processing timestamp ID, 
and links to the panDA workflow IDDS webpages.

\section{DP0.2 wallclock estimates compared with DRP requirements}

Over the course of 150 calendar days in 2021-2022, preliminary Data
Release Production (DRP) processing ran in the iDF on about 20,000
exposures of simulated Rubin data, representing 10-20 full nights of
images, covering 150 tracts (300 square degrees), to typically 5 year
depth (50 returns to the same spot on the sky for each of 6 filters).
This half-percent survey used 2.5M core-hours on a Google cloud compute 
cluster of approximately 4000 cores, each core with (normally) up 
to 14 GB/core RAM available (less typically used) with jobs 
taking typically a few seconds to a fraction of an hour per quanta.  
Small subsets (a few percent) of the processing pipetasks used nodes 
with up to 236 GB/core, and were allowed to run for up to 3 days.  
The input images were approximately 260 TB in size, while the outputs, 
counting transient datasetTypes, expanded to 3.4 PB in an object store.  
After removing transient datasetTypes (e.g. warps), the final storage 
used by DP0.2 was approximately 2.5 PB.  Of the 150 calendar days, 
approximately 60 days were running and the remainder used for 
debugging and development.

Thus, to keep up with expected DRP during the initial stages of the survey
(6-12 days to process 10-20 nights of data), a factor of 5-10 speed up
(or increased core counts or efficiency, or some combination of both) 
are estimated to be needed.   


\appendix
% Include all the relevant bib files.
% https://lsst-texmf.lsst.io/lsstdoc.html#bibliographies
\section{References} \label{sec:bib}
\renewcommand{\refname}{} % Suppress default Bibliography section
\bibliography{local,lsst,lsst-dm,refs_ads,refs,books}

% Make sure lsst-texmf/bin/generateAcronyms.py is in your path
\section{Acronyms} \label{sec:acronyms}
\addtocounter{table}{-1}
\begin{longtable}{p{0.145\textwidth}p{0.8\textwidth}}\hline
\textbf{Acronym} & \textbf{Description}  \\\hline

B & Byte (8 bit) \\\hline
BPS & Batch Production Service \\\hline
CCD & Charge-Coupled Device \\\hline
DC & Data Center \\\hline
DESC & Dark Energy Science Collaboration \\\hline
DM & Data Management \\\hline
DP0 & Data Preview 0 \\\hline
DRP & Data Release Production \\\hline
GB & Gigabyte \\\hline
LSST & Legacy Survey of Space and Time (formerly Large Synoptic Survey Telescope) \\\hline
MB & MegaByte \\\hline
OPS & Operations \\\hline
PB & PetaByte \\\hline
POSIX & Portable Operating System Interface \\\hline
PanDA &  Production ANd Distributed Analysis system \\\hline
RAM & Random Access Memory \\\hline
RTN & Rubin Technical Note \\\hline
S3 & (Amazon) Simple Storage Service  \\\hline
TB & TeraByte \\\hline
bps & bit(s) per second \\\hline
\end{longtable}

% If you want glossary uncomment below -- comment out the two lines above
%\printglossaries


\end{document}
